%\iffalse meta-comment
%
% Copyright (C) 2012 by Maïeul Rouquette 
%
% This file may be distributed and/or modified under the 
% conditions of the LaTeX Project Public License, either 
% version 1.3 of this license or (at your option) any later 
% version. The latest version of this license is in: 
%
% http://www.latex-project.org/lppl.txt 
%
% and version 1.3 or later is part of all distributions of 
% LaTeX version 2003/06/01 or later.
% This work has the LPPL maintenance status "maintained".
%
% \fi

% \iffalse
%\NeedsTeXFormat{LaTeX2e}
%\ProvidesPackage{eleform}[29/09/2012 v1.0 formalism for eledmac]
%<*driver>
\documentclass{ltxdoc}
\usepackage{polyglossia}
\setmainlanguage{english}
\usepackage{eledmac,eledform}
\usepackage{hyperref}
\EnableCrossrefs
\CodelineIndex
\RecordChanges
\begin{document}
\DocInput{eledform.dtx} 
\end{document}
%</driver>
% \fi
% \CheckSum{79}
% \CharacterTable
%  {Upper-case    \A\B\C\D\E\F\G\H\I\J\K\L\M\N\O\P\Q\R\S\T\U\V\W\X\Y\Z
%   Lower-case    \a\b\c\d\e\f\g\h\i\j\k\l\m\n\o\p\q\r\s\t\u\v\w\x\y\z
%   Digits        \0\1\2\3\4\5\6\7\8\9
%   Exclamation   \!     Double quote  \"     Hash (number) \#
%   Dollar        \$     Percent       \%    Ampersand     \&
%   Acute accent  \'     Left paren    \(     Right paren   \)
%   Asterisk      \*     Plus          \+     Comma         \,
%   Minus         \-     Point         \.     Solidus       \/
%   Colon         \:     Semicolon     \;     Less than     \<
%   Equals        \=     Greater than  \>     Question mark \?
%   Commercial at \@     Left bracket  \[     Backslash     \\
%   Right bracket \]     Circumflex    \^     Underscore    \_
%   Grave accent  \`     Left brace    \{     Vertical bar  \|
%   Right brace   \}     Tilde         \~}
%
% \changes{v1.0}{2012/09/26}{First public release}
% \GetFileInfo{eleform.sty}
%\DoNotIndex{\csgdef,\if,\fi,\else,\listgadd}
%
%\title{Eledform extension for eledmac\thanks{This handnote is for version~\fileversion of \textsf{eledform}, last revised \filedate.}}
% \author{Maïeul Rouquette}
% \maketitle
% \begin{abstract}
% The \emph{eledmac} package provides tools to make critical edition of text.
% But it doesn't provide any formalism to note the textual variants: all user have to make is own formalism.
% This package try to provide a formalism
% which distinguishes between the \emph{formal} notation of textual variants and their typographic notation.
% \end{abstract}
% \tableofcontents
%\StopEventually{\PrintChanges}
% \section{Implementation}
% \subsection{Define manuscripts}
% \begin{macro}{\manuscripts@}
% The \cs{manuscripts@} macro is an etoolbox list.
%    \begin{macrocode}
\def\manuscripts@{}
%    \end{macrocode}
% \end{macro}
% \begin{macro}{\manuscripts}
% The \cs{manuscripts} macro only fills the \cs{manuscripts} macro.
%    \begin{macrocode}
\newcommand{\manuscript}[1]{%
    \renewcommand{\do}[1]{\listgadd{\manuscripts@}{##1}}%
    \docsvlist{#1}%
}
%    \end{macrocode}
% \end{macro}
% \subsection{Print the critical notes}
% \begin{macro}{\var}
% The \cs{var} macro is the only public macro, which calls all private macro.
%    \begin{macrocode}
\newcommandx*{\var}[5][1,5,usedefault]{%
%    \end{macrocode}
% First, call the \cs{edtext} macro.
%    \begin{macrocode}
    \edtext{#2}{%
%    \end{macrocode}
% If the \cs{var} macro is called with the optional first argument, 
% we put it into the \cs{lemma} macro.
%    \begin{macrocode}
        \ifstrempty{#1}{}{\lemma{#1}}%
%    \end{macrocode}
% The \cs{varnote@} macro is an Xfootnote macro defined by user, with the \cs{varnote} macro (cf.~\pageref{varnote}).
%    \begin{macrocode}
        \varnote@{%
%    \end{macrocode}
% If the third arg is not empty, we call the \cs{del@} macro, which print the manuscript where the lemma is omitted
%    \begin{macrocode}
            \ifstrempty{#3}{}%
        {\del@{#3}%
%    \end{macrocode}
%  If the third and fourth arg are both not empty, we print the separator bewteen variants.
%    \begin{macrocode}
        \ifstrempty{#4}%
            {}%
            {\varseparator@}%
        }%
%    \end{macrocode}
% And so, we print all the variants which are not omission.
%  \begin{macrocode}
            \var@{#4}%
        }%
%    \end{macrocode}
% Eventually, we add the critical notes which are not for textual criticism.
% \begin{macrocode}
    #5%
    }%
}
%    \end{macrocode}
% \end{macro}
% \begin{macro}{\print@manuscripts}
% The \cs{print@manuscripts} command only print the manuscripts where a variant exists.
%    \begin{macrocode}
\newcommand{\print@manuscript}[1]{%
    \ifinlist{#1}{\manuscripts@}%
        {#1}%
        {\eledmac@warning{Unknew man. #1, p.\the\page@num  ; l.\the\line@num}\underline{unknew man. #1}}%
    }
%    \end{macrocode}
% \end{macro}
% \begin{macro}{\del@}
% The \cs{del@} macro prints the manuscripts where the lemma is omitted,
% and after that, the text to indicate this omission.
%    \begin{macrocode}
\newcommand{\del@}[1]{%
    \renewcommand{\do}[1]{%
        \print@manuscript{##1}%
    }%
    \docsvlist{#1}\manvarseparator@\omittext@%
}
%    \end{macrocode}
% \end{macro}
% \begin{macro}{\var@}
% The \cs{var@} macro loops on the non deletion variants. 
% Except for the first variant, it prints the variant separator.
%    \begin{macrocode}
\newcommand{\var@}[1]{%
    \newif\iffirst%
    \firsttrue%
    \renewcommand{\do}[1]{\iffirst\firstfalse\else\varseparator@\fi{\var@@##1}}%
    \docsvlist{#1}%
    }
%    \end{macrocode}
% \end{macro}
% \begin{macro}{\var@@}
% The \cs{var@@} macro prints manuscripts for a singular variant,
% and after that, this variant.
%    \begin{macrocode}
\newcommand{\var@@}[2]{%
    \renewcommand{\do}[1]{%
        \print@manuscript{##1}%
        }%
    \docsvlist{#1}\manvarseparator@#2%
    }
%    \end{macrocode}
% \end{macro}
% \subsection{Customization}
% \subsubsection{The footnote series}
% \begin{macro}{\varnote@}\label{varnote}
% The \cs{varnote@} macro is just a reference to a critical footnote macro of eledmac.
% The default  is \cs{Afootnote}.
%    \begin{macrocode}
\let\varnote@\Afootnote
%    \end{macrocode}
% \end{macro}
% \begin{macro}{\varseries}
% The \cs{varseries} macro just redefine the reference.
%    \begin{macrocode}
\newcommand{\varseries}[1]{\letcs{\varnote@}{#1footnote}}
%    \end{macrocode}
% \end{macro}
% \subsubsection{Display options}
% \begin{macro}{\new@eledform@custom}
% The \cs{new@eledform@custom} macro has two action :
% \begin{enumerate}
%    \item Define the default value of an option (\cs{option@}.
%    \item Create the command which modify this option (\cs{option}).
% \end{enumerate}
%    \begin{macrocode}
\newcommand{\new@eledform@custom}[2]{%
    \csgdef{#1@}{#2}%
    \expandafter\newcommand\csname#1\endcsname[1]{\csgdef{#1@}{##1}}%
}
%    \end{macrocode}
% \end{macro}
% And so, we can call \cs{new@eledform@custom} to define options for user.
% \begin{macro}{\omittext}
% \begin{macro}{\manvarseparator}
% \begin{macro}{\varseparator}
%    \begin{macrocode}
\new@eledform@custom{omittext}{\emph{omit}}
\new@eledform@custom{manvarseparator}{~}
\new@eledform@custom{varseparator}{\space}
%    \end{macrocode}
% \end{macro}
% \end{macro}
% \end{macro}
% \newpage
% \phantomsection\addcontentsline{toc}{section}{Changes}
% \PrintChanges
% \Finale
\endinput